\section{PHENOBox}
As part of our collaboration with the Vienna Bio Center, we were provided with the so-called PHENOBox as a product environment. The PHENOBox is a flexible solution for plant phenotyping research using image data. During the course of our diploma project, the PHENOBox will be equipped with 2 high-frequency microphones. \cite{czedik-eysenberg_phenobox_2018}

\subsection{Technical Specifications}
The PHENOBox has a height of 75.5 cm, a width of 50 cm, and a length of 82 cm (side) + 4 cm (door handle). It weighs approximately 15 kg. The imaging area is suitable for plants with a maximum shoot size of about 20 cm in width and 40 cm in height. \cite{czedik-eysenberg_phenobox_2018}

The camera can be adjusted in height to achieve a centered position on the shoot for imaging. A Canon SLR camera (700D) is installed in the PHENOBox for easy operation and high image quality. \cite{czedik-eysenberg_phenobox_2018}

The homogeneous illumination of the imaging area is achieved by 7 LED panels, two on each side of the image axis (vertical) and three (horizontal) above the shoot imaging area. \cite{czedik-eysenberg_phenobox_2018}

\section{PHENOPipe}
PhenoPipe is an open-source analysis platform for managing and evaluating samples that complements the hardware of the PhenoBox. It allows for the automation of image processing and data analysis and provides a web interface for managing phenotyping projects. The entire code for PHENOBox and PHENOPipe is located in a public Github repository. \cite{czedik-eysenberg_phenobox_2018}

\textbf{The code consists of the following components:}

\subsection{Phenobox}
Code for the Phenobox itself, which runs on a Raspberry Pi 3.\newline
Captures images and interacts with the Phenopipe server.

\subsection{PhenoPipe Web}
Main server with database and web interface.\newline
Written in Python with Flask and TypeScript with Angular.\newline
Manages experiments, labels, analyses, etc.

\subsection{PhenoPipe Postprocess Server}
A Java-GRPC server for postprocessing analysis data with R scripts.\newline
Setup instructions

\subsection{Phenopipe IAP Server}
A Java-GRPC server for interacting with the IAP (Image Analysis Pipeline).

Although the PHENOPipe has many features that could be potentially useful for our project the complexity and the lack of maintenance could bring more disadvantages than benefits.
