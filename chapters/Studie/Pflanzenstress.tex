\section{Pflanzenstress}
Unter dem Begriff Pflanzenstress versteht man es in der Biologie,dass Pflanzen verschiedenen Faktoren ausgesetzt werden, welche unnatürlich oder kritisch für eine Pflanze sind und somit in dieser Pflanze dann die Freisetzung von verschiedenen Terpenen, Fettsäurederivaten, Benzoiden, Phenylpropanoiden und von Aminosäuren abgeleiteten Stoffwechselprodukten auslösen. \cite{jk2010multiple}

\subsection{verschiedene Auslöser von Pflanzenstress}
Bei allen hier aufgelisteten Auslösern ist zu beachten, dass verschiedene Pflanzengattungen unterschiedlichst auf die folgenden Faktoren reagieren und andere Schwellenwerte haben. Alle diese Stressfaktoren können auch irreparable Schäden an Pflanzen anrichten.
Man kann die Stressfaktoren in drei große Gruppen unterteilen\cite{gaspar2002concepts}:
\begin{itemize}
  \item physiche Faktoren
  \item chemische Faktoren
  \item biotische Faktoren
\end{itemize}

\subsubsection{physiche Stressfaktoren}


\paragraph{Wasserstress} ist ein klassischer physischer Stressfaktor. Unter Wasserstress versteht man es, eine Pflanze einer ungeeigneten Menge an Wasser auszusetzen. Dies bedeuted in der Praxis meist, dass man die Pflanze unterbewässert. Theoretisch kann man natürlich auch die Pflanze überbewässern, was weniger oft Einsatz in der Forschung findet.

\paragraph{Temperaturstress} bedeuted, eine Pflanze durch entweder Kälte- oder Wärmestress zu beinflussen.
Kalte Temperaturen  können die Aufnahme von Wasser und Nährstoffen beeinträchtigen. Dies kann zum Austrocknen(Trockenstress) und Verhungern der Zellen führen. Bei extrem kalten Bedingungen kann die Zellflüssigkeit gefrieren, was zum Tod der Pflanze führt. Zu hohe Temperaturen haben auch negativen Einfluss. Durch zu hohe Temperaturen können Zellproteine abgebaut werden oder Zellwände und -membrane können Schmelzen. Temperaturschwankungen führen also zu Stress und Beinträchtigung der normalen Funktionen einer Pflanze. Auch Temperaturstress ist ein klassischer physischer Faktor.


\paragraph{Strahlung, Wind} und weitere Faktoren können auch natürlich auftreten werden aber in dieser Arbeit keine Verwendung finden 

\subsubsection{chemische Stressfaktoren}

\paragraph{Toxine} kommen in der Natur und in der Agrarwelt oft vor. Diese können zum Beispiel über verschmutztes Grundwasser aufgenommen werden und die natürliche Funktion einer Pflanze stark einschränken. Oft werden Toxine auch in Form von Luftverschmutzung von Pflanzen aufgenommen.

\paragraph{Pestizide und Fungizide} finden oft in der Landwirtschaft Einsatz. Gerade deshalb sollte nicht vergessen werden, dass auch diese starken chemischen Stress auslösen können.

\subsubsection{Biotische Stressfaktoren}

\paragraph{Krankheiten, Pilze und Viren} sind die am häufigsten autretenden boitischen Stressfaktoren indem diese eine Pflanze befallen und diese schwächen.

% METHODIK

\subsection{Methodik zur Erfassung des Pflanzenstresses}

Pflanzenstress wird schon länger erforscht, wobei sich die meisten Messmethodiken auf Bilderkennung, Thermalbilder, 3D Scans und auf Hyperspektralbilder zurückgreifen. Es gibt auch weitere Möglichkeiten den Pflanzenstress zu erkennen, jedoch sind das die aktuell am häufigsten verwendeten in der Pflanzenpathologie.
\cite{behmann2015detektion}\cite{messtechniken}

\subsubsection{RGB Bildanalyse}
Durch eine digitale RGB-Farbbildgebung in Verbindung mit einer automatischen, analytischen Software ist eine Extraktion von verschiedenen Merkmalen möglich. Hochauflösende kinetische Messungen werden für eine eingehende Analyse der Pflanzenmorphologie, Architektur und zur Extraktion von Farbindexmerkmalen verwendet.

\subsubsection{3D-Laserscans}
 Ein Laserstrahl wird von der Pflanze reflektiert und von einer Kamera erfasst. Ein automatisierter Phänotypisierungsprozess beinhaltet also das Scannen der Pflanze, die Erstellung eines 3D-Modells, die Erhebung von Pflanzenparametern und die anschließende Analyse. Auf der Grundlage der vernetzten Modelle bietet die automatische Datenanalyse die Berechnung einer breiten Palette von morphologischen Parametern.\cite{vandenberghe2018make}

Diese Technologien und Verfahren ermöglichen eine detaillierte Erfassung und Analyse der Pflanzenmorphologie und -struktur, was für die Erforschung von Pflanzenreaktionen auf verschiedene Stressbedingungen wichtig ist.

\subsubsection{PAM chlorophyll fluorescence}
Die Chlorophyllfluoreszenz-Analyse, insbesondere mittels Puls-Amplituden-Modulation (PAM), ist eine bewährte Methode zur Beurteilung der Photosyntheseleistung und des physiologischen Zustands von Pflanzen, einschließlich des Stresses, dem sie ausgesetzt sind.

Zuerst wirden die Chlorophyllmoleküle in den Pflanzenzellen mittels Lichtmolekülen angeregt. Im angerregten Zustand emittieren die Chlorophyllmoleküle Licht. Die Intensität dieses Lichtes kann dann gemessen und verarbeitet werden.
Änderungen in den Fluoreszenzparametern können auf Stressbedingungen hindeuten, unter denen die Pflanze steht.

Diese Methode ist nicht-invasiv und ermöglicht eine schnelle und genaue Einschätzung des physiologischen Zustands von Pflanzen, was sie zu einem wertvollen Werkzeug für die Pflanzenphänotypisierung und die Bewertung von Pflanzenstress macht.

\subsubsection{LWIR thermal imaging}

\subsubsection{Hyperspectral imaging (VNIR and SVIR)}
